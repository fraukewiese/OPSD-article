\documentclass[article]{elsarticle}

\usepackage{lineno,hyperref}
\modulolinenumbers[5]

\journal{Journal of \LaTeX\ Templates}

%%%%%%%%%%%%%%%%%%%%%%%
%% Elsevier bibliography styles
%%%%%%%%%%%%%%%%%%%%%%%
%% To change the style, put a % in front of the second line of the current style and
%% remove the % from the second line of the style you would like to use.
%%%%%%%%%%%%%%%%%%%%%%%

%% Numbered
%\bibliographystyle{model1-num-names}

%% Numbered without titles
%\bibliographystyle{model1a-num-names}

%% Harvard
%\bibliographystyle{model2-names.bst}\biboptions{authoryear}

%% Vancouver numbered
%\usepackage{numcompress}\bibliographystyle{model3-num-names}

%% Vancouver name/year
%\usepackage{numcompress}\bibliographystyle{model4-names}\biboptions{authoryear}

%% APA style
%\bibliographystyle{model5-names}\biboptions{authoryear}

%% AMA style
%\usepackage{numcompress}\bibliographystyle{model6-num-names}

%% `Elsevier LaTeX' style
\bibliographystyle{elsarticle-num}
%%%%%%%%%%%%%%%%%%%%%%%

\begin{document}

\begin{frontmatter}

\title{Open Power System Data - Frictionless data for modelling}
\iffalse
%% Group authors per affiliation:
\author{Elsevier\fnref{myfootnote}}
\address{Radarweg 29, Amsterdam}
\fntext[myfootnote]{Since 1880.}


%% or include affiliations in footnotes:
\author[mymainaddress,mysecondaryaddress]{Elsevier Inc}
\ead[url]{www.elsevier.com}

\author[mysecondaryaddress]{Global Customer Service\corref{mycorrespondingauthor}}
\cortext[mycorrespondingauthor]{Corresponding author}
\ead{support@elsevier.com}

\address[mymainaddress]{1600 John F Kennedy Boulevard, Philadelphia}
\address[mysecondaryaddress]{360 Park Avenue South, New York}
\fi
\begin{abstract}
\begin{itemize}
    \item Power System Modelling is heavily depending on high quality data
    \item Transparency of it highly depends on the question if input data is provided
    \item Main challenges are to (i) use just openly available data, (ii) to preprocess the data in a transparent way (iii) to verify the data
     \item This article presents an approach to tackle these challenges
    %\item This article presents an approach to tackle these challenges, summarizes remaining legal issues and concludes suggestions on the requirements to progress on energy data provision
\end{itemize}
\end{abstract}

\begin{keyword}
\sep \sep 
\end{keyword}

\end{frontmatter}

\linenumbers

\section*{NOTES}
\begin{itemize}
    \item Irgendwo noch Legal Status, zumindest kurz erwaehnen: Legal Context: MIT for the scripts, but licensing the data itselfs is facing some problems, explain the lawyer assessments and communication with the original data owners
\end{itemize}

\section{Introduction (Frauke)}
(Background single slide OPSD)

\subsection{Need for quantitative power sector modeling}

\subsection{Extensive input data requirements}
in particular generation capacities and time series of load and RES (wie zufällig sind das die OPSD-Datenpakete) 

\subsection{Availability of data}
A lot of data is available - but dispersed, poorly documented and ill-formatted 

% Additional notes for problems of input data:
% The problem of input data
%     \begin{itemize}
%         \item availability of open data
%         \item amount of data and sources - mix of many different sources required, difficult to find with e.g. google
%         \item manual download of different files and merging of different file-formats, formatting
%         \item reproducible preprocessing
%         \item flexible preprocessing with the possibility to update along with the original data
%         \item clearing up, verification/validation of input data ( inconsistencies, errors, gaps in data)
%         \item visualisation?
%         \item unclarity of legal issues
%     \end{itemize}
    
\section{OPSD: frictionless data for modelling (Ingmar)}
\subsection{General idea}
gather "official" data in one place, foramt it nicely, quality checks / improvements; concept of data packages 

\subsection{IT concept}
"leightweight", open software, GitHub etc., data packages, scripts including documentation, storage of original data sources, version-control, metadata (json)

\subsection{Data standards}
Frictionless data (OKF), evtl.in 2.3 mit reinziehen)

\subsection{Some (careful) data modifications}
wenn möglich und sinnvoll, hier high-level Beschreibung des Korrekturen in den einzelnen Datenpaketen Approach of OPSD / Verification/Validation - trade-off with changing the original input data too much: Concept of error marking IT-concept: 

\section{Results: Five data packages}
\subsection{Overview (?)}
3.1 - 3.5 jeweils kurze Beschreibung der packages (Vertextung unserer Folien vom Abschlussworkshop), idealerweise mit einer Beispiel-Tabelle oder noch besser Visualisierung; ggf auch mit geographischer Coverage stand heute (oder Stand Ende OPSD), so wie auf dem Workshop 
\subsection{Conventional power plants (Wolf)}
\subsection{National generation capacity (Wolf)}
\subsection{Renewable power plants (Frauke)}
\subsection{Time series (Jonathan)}
The time series data package contains highly granular (30/15/60 minutes) time series data for 35 European countries, including the following variables:
\begin{itemize}
    \item Power consumption (load)
    \item Power generation by varaiable renewables (wind and solar) and share of production capacity 
    \item Day-ahead power prices
\end{itemize}
Data are extracted from a number of sources, including national TSO's websites, ENTSO-E's "Monthly statistics data collection" as well as ENTSO-E Transparency.
Power system modelling usually requires time series data to be complete for for the time period under consideration. Many data sources however 


\subsection{Weather data (Frauke)}
        
\section{Discussion and Outlook (Lion)}
\begin{itemize}
    \item Wolf: hier insbesondere auf laufende Erweiterungen etc. eingehen, möglicherweise auch auf Zielgruppen etc. 
    \item Statistics on how the data is applied, how many users etc., maybe categorize the applications
    \item MaybesShow verification/validation results of bottom-up power plant data with national statistics
    \item The approach of OPSD is reproducible, flexible, possible to update and widely apply
    \item quality of the data, room for improvement: what specific?
    \item maintenance work is required
    \item Procedure cannot be applied for all data required for energy system modelling, e.g. for grid stuff probably, there database approaches are also required
    \item tools and knowledge for reproducible data processing and metadata storage is there but data education (documentation, organisation) for energy system modellers is a bottleneck while important
    \item legal issue is central, examples of improvement, but data knowledge and knowledge of open licences for original data owners is lacking
\end{itemize}

\section*{References}

\bibliography{mybibfile}

\end{document}